%%%%%%%%%%%%%%%%%%%%%%%%%%%%%%%%%%%%%%%%%%%%%%%%%%%%%%%%%%%%%%%%%%%%%%
%
%     This is a LaTex template to be used for a PhD thesis in the
%     Department of Physics at Carnegie Mellon University. It is 
%     recommended to input individual chapters, appendices, etc
%     as separate files. To avoid having to provide multiple files
%     with this template, simple examples for such separate files
%     are included with this template.
%                                                       July 2010
%
%     For comments contact: Manfred Paulini <paulini@cmu.edu>
%
%%%%%%%%%%%%%%%%%%%%%%%%%%%%%%%%%%%%%%%%%%%%%%%%%%%%%%%%%%%%%%%%%%%%%%
%
\documentclass[12pt,twoside]{report}
%
% ====================================================================
%     Include you favorite packages 
%     Here only basic packages used for LaTex with PS output file
% ====================================================================
%
\usepackage{color}
\usepackage{graphicx}                   
\DeclareGraphicsExtensions{.eps,.ps}               
%
% ====================================================================
%     Some formatting suggestions
% ====================================================================
%
\topmargin -0.3in
\oddsidemargin 0.5in
\evensidemargin 0.5in
\textheight 8.5in
\textwidth 6.0in
%
% ====================================================================
%     The following commands tent to keep LaTex happier in the 
%     placement of figures, tables, etc
% ====================================================================
%
\renewcommand{\textfraction}{0.0}
\renewcommand{\floatpagefraction}{0.0}
\renewcommand{\topfraction}{1.0}
\renewcommand{\bottomfraction}{1.0}
\setcounter{topnumber}{9}
\setcounter{bottomnumber}{9}
\setcounter{totalnumber}{9}
% 
% ====================================================================
%     Some of the things we need for the title page
% ====================================================================
%
\author{\\
	\\
	\\
	by \\
	\\
      	Matthew Klingensmith \\
	\\
	\\
	\\
	\\
	\\
        Submitted in partial fulfillment of the \\
        requirements for the degree of \\
        Doctor of Philosophy \\
        at \\
        Carnegie Mellon University \\
        Robotoics Institute\\
        Pittsburgh, Pennsylvania \\
	\\
        \\
	Advised by Professors \\
	Siddharta S. Srinivasa \\
	Michael Kaess
	\\
	\\
}

\title{\Huge\textbf{
Thesis Proposal: Kinematically Constrained Dense 3D SLAM
}}

\date{\today}

%
% ====================================================================
%     Input your definition in file definitions.tex here. 
%     A simple example of the content of such a file is given below.
% ====================================================================
%
%\input{definitions}
%
%%%%%%%%%%%%%% Example for content of file definitions.tex %%%%%%%%%%%
\def\ra    {\rightarrow}
\def\ul    {\underline} 
\def\mevcc {\ifmmode {\mathrm MeV}/c^2 \else MeV$/c^2$\fi}
\def\BsJpsiPhi {\ensuremath{\Bs \ra J/\psi\,\phi}}
%%%%%%%%%%%%%%%%%%%%%%%%%%%%%%%%%%%%%%%%%%%%%%%%%%%%%%%%%%%%%%%%%%%%%%
%
\begin{document} 

\thispagestyle{empty}

\maketitle

\newpage

\chapter{Introduction}

\begin{itemize}
    \item Dense 3D reconstruction is useful in many robotics contexts such as
    scene understanding and planning.
    \item Typical dense 3D reconstruction techniques assume that nothing is
    known about how the sensor moves aside from weak priors on the magnitude of
    the motion, or else use inertial measurment units.
    \item For this reason, dense reconstruction is treated as an unconstrained
    SLAM problem, where the pose of the sensor is mostly inferred from the
    change in the map.
    \item This leads to significant pose drift.
    \item Robots, however, often have sensors mounted on actuated linkages, such
    as necks or arms. These provide very strong priors on the motion of the
    sensor due to their kinematics.
    \item Consider the case of a robot with a hand-mounted 2D or 3D sensor. The
    robot scans the scene and produces a 3D reconstruction.
    \item With perfectly known robot kinematics, this is easy, and the problem
    becomes one of simple mapping, rather than SLAM.
    \item However, even kinematically constrained robotic linkages may have
    significant noise and inaccuracies. Hysteresis from relative joint encoders
    leads to inaccuracies that compound over time, and on some systems, cable
    slack and nonrigid deformation prevent us from knowing the robot kinematics
    with certainty.
    \item We can correct for this uncertainty using the sensor mounted on the
    arm. 
    \item Rather than estimating the pose of the sensor directly, we can
    estimate the joint angles of the robot, and hopefully arrive at a better
    pose estimate this way.
    \item In 3D, the solution is related to dense 3D odometry. We can use
    kinematically-constrained gradient descent to produce a pose estimate.
    \item In 2D, the solution is related to visual odometry and visual servoing.
    We can compute feature tracks in the image, and then use the robot's image
    Jacobian to compute a change in joint angles.
    
\end{itemize}

\chapter{Framework}
\section{3D Mapping and Localization}
\begin{itemize}
    \item Discuss occupancy grids, point clouds, and the Truncated Signed
    Distance field
    \item A TSDf is a good idea to use in this context, because it implicitly
    contains gradient information.
    \item Kinect Fusion gracefully localizes and maps simultaneously using the
    TSDf as an objective function that informs the motion of the camera.
\end{itemize}

\section{Robot Kinematics and Arm Localization}
\begin{itemize}
    \item Robot linkages have well-defined kinematic constraints: joint limits,
    and rigid forward kinematics.
    \item When joint angles are known perfectly, the pose of any rigidly
    attached object on the body (including the sensor) is known.
    \item Nonrigid link deformation, hysterisis, and cable stretch can lead to
    incorrect forward kinematics. On some robots (such as \textit{Baxter} and
    some models of the \textit{Barret WAM}), this error is pronounced.
    \item When an external sensor is used (disconnected from the robot), one can
    track the configuration of the robot accurately.
    \item Tracking the configuration of the robot with an \textit{attached}
    sensor is less well studied -- though all visual servoing techniques
    implicitly have to solve this problem.
    \item A Kinect-Fusion-like algorithm can be used to simultaneously track the
    robot's configuration and map its environment.
\end{itemize}

\chapter{Research Questions}
\begin{itemize}
    \item In robotic arms, what is the typical uncertainty we should expect, and
    where does it come from?
    \item What benefit do we get out of using kinematically constrained dense
    odometry versus unconstrained dense odometry?
    \item Robot kinematics are underconstrained by sensor measurements, what
    pitfalls does this cause? How do we deal with this?
\end{itemize}

\chapter{Timeline}
\begin{itemize}
    \item Develop theory for kinematically constrained 3D visual odometry.
    \item Run experiments in 2D simulation to verify theory
    \item Run experiments in 3D simulation to verify theory
    \item Get 3D dense odometry working on the ADA arm with fake joint noise
    \item Attach a sensor to HERB's arm and do it with real noise.
    \item Attach a sensor to Abhinav's Baxter robot
    \item Extend theory to 2D visual odometry, and use Baxter's hand cam to
    localize it.
\end{itemize}
\end{document}


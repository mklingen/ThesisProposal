%%%%%%%%%%%%%%%%%%%%%%%%%%%%%%%%%%%%%%%%%%%%%%%%%%%%%%%%%%%%%%%%%%%%%%
%
%     This is a LaTex template to be used for a PhD thesis in the
%     Department of Physics at Carnegie Mellon University. It is 
%     recommended to input individual chapters, appendices, etc
%     as separate files. To avoid having to provide multiple files
%     with this template, simple examples for such separate files
%     are included with this template.
%                                                       July 2010
%
%     For comments contact: Manfred Paulini <paulini@cmu.edu>
%
%%%%%%%%%%%%%%%%%%%%%%%%%%%%%%%%%%%%%%%%%%%%%%%%%%%%%%%%%%%%%%%%%%%%%%
%
\documentclass[12pt,twoside]{report}
%
% ====================================================================
%     Include you favorite packages 
%     Here only basic packages used for LaTex with PS output file
% ====================================================================
%
\usepackage{color}
\usepackage{graphicx}                   
\DeclareGraphicsExtensions{.eps,.ps}               
%
% ====================================================================
%     Some formatting suggestions
% ====================================================================
%
\topmargin -0.3in
\oddsidemargin 0.5in
\evensidemargin 0.5in
\textheight 8.5in
\textwidth 6.0in
%
% ====================================================================
%     The following commands tent to keep LaTex happier in the 
%     placement of figures, tables, etc
% ====================================================================
%
\renewcommand{\textfraction}{0.0}
\renewcommand{\floatpagefraction}{0.0}
\renewcommand{\topfraction}{1.0}
\renewcommand{\bottomfraction}{1.0}
\setcounter{topnumber}{9}
\setcounter{bottomnumber}{9}
\setcounter{totalnumber}{9}
% 
% ====================================================================
%     Some of the things we need for the title page
% ====================================================================
%
\author{\\
	\\
	\\
	by \\
	\\
      	Matthew Klingensmith \\
	\\
	\\
	\\
	\\
	\\
        Submitted in partial fulfillment of the \\
        requirements for the degree of \\
        Doctor of Philosophy \\
        at \\
        Carnegie Mellon University \\
        Robotoics Institute\\
        Pittsburgh, Pennsylvania \\
	\\
        \\
	Advised by Professors \\
	Siddharta S. Srinivasa \\
	Michael Kaess
	\\
	\\
}

\title{\Huge\textbf{
Thesis Proposal: A Volumetric Pipeline for Robotic Manipulation 
}}

\date{\today}

%
% ====================================================================
%     Input your definition in file definitions.tex here. 
%     A simple example of the content of such a file is given below.
% ====================================================================
%
%\input{definitions}
%
%%%%%%%%%%%%%% Example for content of file definitions.tex %%%%%%%%%%%
\def\ra    {\rightarrow}
\def\ul    {\underline} 
\def\mevcc {\ifmmode {\mathrm MeV}/c^2 \else MeV$/c^2$\fi}
\def\BsJpsiPhi {\ensuremath{\Bs \ra J/\psi\,\phi}}
%%%%%%%%%%%%%%%%%%%%%%%%%%%%%%%%%%%%%%%%%%%%%%%%%%%%%%%%%%%%%%%%%%%%%%
%
\begin{document} 

\thispagestyle{empty}

\maketitle

\newpage

\chapter{Introduction}

\begin{itemize}
    \item Robots need to be able to build models of the world to understand it
    \item There are many kinds of competing models (sparse point clouds,
    surfaces, volumetric, semantic)
    \item Volumetric models have many advantages, such as resistance to noise,
    high quality, dense structure, and more.
    \item Different stages of the typical robot manipulation pipeline use
    different models.
    Reconstruction often uses point clouds, and geometric surfaces are often
    fitted to the point clouds and planned for, without regard to uncertainty.
    \item The typical pipeline suffers from ``early cutoffs'' of uncertainty.
    Grasp planners fail to care about unknown or free space. Motion planning
    suffers from the same problems.
    \item A complete pipeline is needed which uses the \emph{same} model
    throughout the pipeline, and propagates uncertainty to do so.
    \item I believe a volumetric representation of the scene is best.
\end{itemize}

\chapter{Framework}

\section{3D Reconstruction and State Estimation}
\begin{itemize}
    \item Space carving + occupancy grids
    \item TSDF reconstruction, Chisel
    \item Arm tracking. Arm mounted sensor? 
\end{itemize}

\section{Scene Segmentation}
\begin{itemize}
    \item Direct segmentation of the volumetric scene
    \item Graph-cut based approach?
    \item Learning ``Objectness'' from example.
\end{itemize}
\section{Object Recognition}
\begin{itemize}
    \item Once the scene is segmented, can we recognize objects directly from
    the volumetric data?
    \item Possible features to use for segmented TSDF data?
    \item Once an object is recognized, what's the best way to fit it to the
    data?
\end{itemize}
\section{Motion Planning}
\begin{itemize}
    \item To get the robot to move its arms around the scene, we can use the
    volumetric data structure \emph{directly} in the planner.
    \item CHOMP and/or trajopt can be directly applied to the volumetric data
    structure.
\end{itemize}
\section{Grasping and Manipulation\\
                of Volumetric Objects}
\begin{itemize}
    \item Once objects are segmented, we can use the volumetric data structure
    to inform grasping.
    \item Compute force closure using distances/contact normals.
    \item Avoid unknown space, or perhaps garuntee bounds on grasping?
    \item Stretch goal: learn the model through touch?
\end{itemize}

\chapter{Research Questions}
\begin{itemize}
    \item What does propogating the \emph{same} volumetric model throughout the
    pipeline give us over a more heterogeneous approach?
    \item How can each part of the pipeline benefit from a volumetric model
    over, say, surfaces or point clouds?
    \item How can knowledge of the sensor's uncertainty be incorporated into a
    volumetric distance field model?
    \item What are the drawbacks/limitations of a volumetric model?
\end{itemize}


\chapter{Timeline}


% ====================================================================
%     Generates one blank page before the 'Abstract'
% ====================================================================
% 
% \thispagestyle{empty} \cleardoublepage
% 
% % ====================================================================
% %     Input your abstract in file abstract.tex here. 
% %     A simple example of the content of such a file is given below.
% % ====================================================================
% %
% %\include{abstract}
% %
% %%%%%%%%%%%%%% Example for content of file abstract.tex %%%%%%%%%%%%%%
% \begin{abstract}
% 
%   This is the most important discovery made in the history of modern
%   physics. We discuss the detailed measurement that solved the mystery
%   of the universe.  This is the most important discovery made in the
%   history of modern physics. We discuss the detailed measurement that
%   solved the mystery of the universe.  This is the most important
%   discovery made in the history of modern physics. We discuss the
%   detailed measurement that solved the mystery of the universe.
% 
% \end{abstract}
% %%%%%%%%%%%%%%%%%%%%%%%%%%%%%%%%%%%%%%%%%%%%%%%%%%%%%%%%%%%%%%%%%%%%%%
% 
% \thispagestyle{empty} \cleardoublepage
% 
% \pagenumbering{roman}
% 
% % ====================================================================
% %     Input your acknowledgments in file acknowledgments.tex here. 
% %     A simple example of the content of such a file is given below.
% % ====================================================================
% %
% %\include{acknowledgments}
% %
% %%%%%%%%%%%%%% Example for content of file acknowledgments.tex %%%%%%
% \section*{Acknowledgments}
% 
% I would like to express my profound gratitude to my advisor and I thank
% everybody else, too.  I would like to express my profound gratitude to
% my advisor and I thank everybody else, too.  I would like to express my
% profound gratitude to my advisor and I thank everybody else, too.  I
% would like to express my profound gratitude to my advisor and I thank
% everybody else, too.  I would like to express my profound gratitude to
% my advisor and I thank everybody else, too.
% %%%%%%%%%%%%%%%%%%%%%%%%%%%%%%%%%%%%%%%%%%%%%%%%%%%%%%%%%%%%%%%%%%%%%%
% 
% \tableofcontents 
% 
% \listoftables
% 
% \listoffigures
% 
% \clearpage 
% 
% \pagenumbering{arabic}
% 
% % ====================================================================
% %     Include chapter 1 in file chap_01.tex here. 
% %     A simple example of the content of such a file is given below.
% % ====================================================================
% %
% %\include{chap_01}
% %
% %%%%%%%%%%%%%% Example for content of file chap_01.tex %%%%%%%%%%%%%%%
% \chapter{Test 1 - Chapter}
% 
% This is a test, nothing but a test, a test, just a test and a test.
% This is a test, nothing but a test, a test, just a test and a test.
% This is a test, nothing but a test, a test, just a test and a test.
% This is a test, nothing but a test, a test, just a test and a test.
% This is a test, nothing but a test, a test, just a test and a test.
% This is a test, nothing but a test, a test, just a test and a test.
% This is a test, nothing but a test, a test, just a test and a test.
% This is a test, nothing but a test, a test, just a test and a test.
% This is a test, nothing but a test, a test, just a test and a test.
% 
% \section{Test 1 - Section}
% 
% This is a test, nothing but a test, a test, just a test and a test.
% This is a test, nothing but a test, a test, just a test and a test.
% This is a test, nothing but a test, a test, just a test and a test.
% This is a test, nothing but a test, a test, just a test and a test.
% This is a test, nothing but a test, a test, just a test and a test.
% This is a test, nothing but a test, a test, just a test and a test.
% This is a test, nothing but a test, a test, just a test and a test.
% This is a test, nothing but a test, a test, just a test and a test.
% 
% This is a test, nothing but a test, a test, just a test and a test.
% This is a test, nothing but a test, a test, just a test and a test.
% This is a test, nothing but a test, a test, just a test and a test.
% This is a test, nothing but a test, a test, just a test and a test.
% This is a test, nothing but a test, a test, just a test and a test.
% 
% % ====================================================================
% %     Uncomment for some simple examples on how to create tables
% %     or include figures.
% % ====================================================================
% %
% %%%%%%%%%%%%%%%%%%%%%%%%%%%%%%%%%%%%%%%%%%%%%%%%%%%%%%%%%%%%%%%%%%%%%%
% %%%%% ----------------------------------------------------------------
% \begin{table}[tbh]
% \caption{Summary of the most important results.}
% %
% \begin{center}
%   \begin{tabular}{|c|cc|}
%     \hline
%     A   & B   & C  \\
%     \hline
%     1   & 2   & 3   \\
%     11  & 22  & 33  \\
%     1   & 2   & 3   \\
%     \hline
%   \end{tabular}
% \end{center}
% \label{Tab:table1}
% \end{table}
% %%%%% ----------------------------------------------------------------
% 
% %%%%% ----------------------------------------------------------------
% %\begin{figure}[tb]
% %\begin{center}
% %\includegraphics[width=0.5\textwidth]{my_result.eps}
% %\caption{
% %The Nobel prize awarded measurement.
% %}
% %\label{Fig:nobel1}
% %\end{center}
% %\end{figure}
% %%%%% ----------------------------------------------------------------
% %%%%%%%%%%%%%%%%%%%%%%%%%%%%%%%%%%%%%%%%%%%%%%%%%%%%%%%%%%%%%%%%%%%%%%
% 
% \subsection{Test 1 - SubSection}
% 
% This is a test, nothing but a test, a test, just a test and a test.
% This is a test, nothing but a test, a test, just a test and a test.
% This is a test, nothing but a test, a test, just a test and a test.
% This is a test, nothing but a test, a test, just a test and a test.
% This is a test, nothing but a test, a test, just a test and a test.
% This is a test, nothing but a test, a test, just a test and a test.
% This is a test, nothing but a test, a test, just a test and a test.
% This is a test, nothing but a test, a test, just a test and a test.
% 
% This is a test, nothing but a test, a test, just a test and a test.
% This is a test, nothing but a test, a test, just a test and a test.
% This is a test, nothing but a test, a test, just a test and a test.
% This is a test, nothing but a test, a test, just a test and a test.
% This is a test, nothing but a test, a test, just a test and a test.
% This is a test, nothing but a test, a test, just a test and a test.
% This is a test, nothing but a test, a test, just a test and a test.
% This is a test, nothing but a test, a test, just a test and a test.
% 
% \subsubsection{Test 1 - SubSubSection}
% 
% This is a test, nothing but a test, a test, just a test and a test.
% This is a test, nothing but a test, a test, just a test and a test.
% This is a test, nothing but a test, a test, just a test and a test.
% This is a test, nothing but a test, a test, just a test and a test.
% This is a test, nothing but a test, a test, just a test and a test.
% This is a test, nothing but a test, a test, just a test and a test.
% This is a test, nothing but a test, a test, just a test and a test.
% This is a test, nothing but a test, a test, just a test and a test.
% 
% This is a test, nothing but a test, a test, just a test and a test.
% This is a test, nothing but a test, a test, just a test and a test.
% This is a test, nothing but a test, a test, just a test and a test.
% This is a test, nothing but a test, a test, just a test and a test.
% This is a test, nothing but a test, a test, just a test and a test.
% This is a test, nothing but a test, a test, just a test and a test.
% This is a test, nothing but a test, a test, just a test and a test.
% This is a test, nothing but a test, a test, just a test and a test.
% 
% % ====================================================================
% %     Include chapter 2 in file chap_02.tex here. 
% %     A simple example of the content of such a file is given below.
% % ====================================================================
% %
% %\include{chap_02}
% %
% %%%%%%%%%%%%%% Example for content of file chap_02.tex %%%%%%%%%%%%%%%
% \chapter{Test 2 - Chapter}
% 
% And here we go again.  And here we go again.  And here we go again.  And
% here we go again.  And here we go again.  And here we go again.  And
% here we go again.  And here we go again.  And here we go again.  And
% here we go again.  And here we go again.  And here we go again.
% 
% \section{Test 2 - Section}
% 
% It's a kind of magic.  A kind of magic.  One dream one soul, one prize.
% One goal, one golden glance of what should be.  It's a kind of magic.
% It's a kind of magic.  A kind of magic.  One dream one soul, one prize.
% One goal, one golden glance of what should be.  It's a kind of magic.
% It's a kind of magic.  A kind of magic.  One dream one soul, one prize.
% One goal, one golden glance of what should be.  It's a kind of magic.
% It's a kind of magic.  A kind of magic.  One dream one soul, one prize.
% One goal, one golden glance of what should be.  It's a kind of magic.
% It's a kind of magic.  A kind of magic.  One dream one soul, one prize.
% One goal, one golden glance of what should be.  It's a kind of magic.
% It's a kind of magic.  A kind of magic.  One dream one soul, one prize.
% One goal, one golden glance of what should be.  It's a kind of magic.
% 
% \subsection{Test 2 - SubSection}
% 
% I think it's stupid and sad that everything turned out so bad \dots
% I think it's stupid and sad that everything turned out so bad \dots
% I think it's stupid and sad that everything turned out so bad \dots
% I think it's stupid and sad that everything turned out so bad \dots
% I think it's stupid and sad that everything turned out so bad \dots
% I think it's stupid and sad that everything turned out so bad \dots
% %%%%%%%%%%%%%%%%%%%%%%%%%%%%%%%%%%%%%%%%%%%%%%%%%%%%%%%%%%%%%%%%%%%%%%
% 
% 
% % ====================================================================
% %     To include references, it is recommended to use BibTex. A simple 
% %     example of including references in a file bib.tex is given.
% % ====================================================================
% %
% %\include{bib}
% %
% %%%%%%%%%%%%%% Example for content of file bib.tex %%%%%%%%%%%%%%%
% \begin{thebibliography}{99}
% 
% %\cite{Ref:Einstein_1935}
% \bibitem{Ref:Einstein_1935}
%   A.~Einstein, B.~Podolsky and N.~Rosen,
%   ``Can quantum mechanical description of physical reality be considered
%   complete?,''
%   Phys.\ Rev.\  {\bf 47}, 777 (1935).
% 
% %\cite{Ref:Fermi_1950}
% \bibitem{Ref:Fermi_1950}
%   E.~Fermi,
%   ``High-energy nuclear events,''
%   Prog.\ Theor.\ Phys.\  {\bf 5}, 570 (1950).
% 
% \end{thebibliography}
% %%%%%%%%%%%%%%%%%%%%%%%%%%%%%%%%%%%%%%%%%%%%%%%%%%%%%%%%%%%%%%%%%%%%%%

\end{document}

